\begin{frame}{Model Descriptions}

This example illustrates how to use functions in the dynr package to fit
a regime-switching linear ODE model of the form:

\(\frac{d\eta_1(t)}{dt} = -r_1\eta_1(t) + a_{12}[\eta_2(t)-\eta_1(t)]\)

\(\frac{d\eta_2(t)}{dt} = -r_2[\eta_2(t)-\eta_{20}] + a_{21}[\eta_2(t)-\eta_1(t)]\)

\begin{itemize}
\itemsep1pt\parskip0pt\parsep0pt
\item
  if \(S_t=1\): \(r_1\) and \(r_2\) are freely estimated; \(a_{12}\)
  \(=\) \(a_{21}\) \(=\) 0;
\item
  if \(S_t=2\): \(r_1\) \(=\) \(r_2\) \(=\) 0; \(a_{12}\) and \(a_{21}\)
  \(=\) are freely estimated.
\item
  The transition between regimes is governed by a transition matrix with
  elements:
  \(\begin{bmatrix} p_{11} & 1 - p_{11}\\ p_{21} & 1 - p_{21} \end{bmatrix}\),
  where \(p_{11}\) and \(p_{21}\) are functions of the covariates
  \(x_{1,it}\) and \(x_{2,it}\) as:
\end{itemize}

\(p11 = \frac{\exp(a_0 + a_1 + a_3*x_{2,it}+a_5*x_{1,it})}{exp(0)+exp(a_0 +a_1 + a_3*x_{2,it}+a_5*x_{1,it})}\)

\(p21 = \frac{\exp(a_0 + a_2*x_{2,it}+a_4*x_{1,it})}{exp(0)+exp(a_0 + a_2*x_{2,it}+a_4*x_{1,it})}.\)

\end{frame}

\begin{frame}[fragile]{Load library and format data}

\begin{Shaded}
\begin{Highlighting}[]
\KeywordTok{require}\NormalTok{(dynr)}
\end{Highlighting}
\end{Shaded}

\begin{verbatim}
## Loading required package: dynr
## Loading required package: numDeriv
## Loading required package: ggplot2
## Loading required package: reshape2
## Loading required package: plyr
\end{verbatim}

\begin{Shaded}
\begin{Highlighting}[]
\NormalTok{thedata =}\StringTok{ }\KeywordTok{read.table}\NormalTok{(}\StringTok{'/Users/symiin/Dropbox/BREKFIS/dynr/data/New2CovModel1T1000n20batch1ODEsimData.txt'}\NormalTok{)}
\NormalTok{thedata$V6 =}\StringTok{ }\KeywordTok{as.numeric}\NormalTok{(thedata$V6)}
\NormalTok{data <-}\StringTok{ }\KeywordTok{dynr.data}\NormalTok{(thedata, }\DataTypeTok{id=}\StringTok{"V1"}\NormalTok{, }\DataTypeTok{time=}\StringTok{"V2"}\NormalTok{,}\DataTypeTok{observed=}\KeywordTok{paste0}\NormalTok{(}\StringTok{'V'}\NormalTok{, }\DecValTok{3}\NormalTok{:}\DecValTok{4}\NormalTok{), }
                  \DataTypeTok{covariates=}\KeywordTok{paste0}\NormalTok{(}\StringTok{'V'}\NormalTok{, }\DecValTok{5}\NormalTok{:}\DecValTok{6}\NormalTok{))}
\end{Highlighting}
\end{Shaded}

\end{frame}

\begin{frame}[fragile]{Model specification}

\begin{itemize}
\itemsep1pt\parskip0pt\parsep0pt
\item
  The parameters to be optimized are:
  \([\log(r_1), \log(r_2), \log(a_{12}), \log(a_{21}), log(\sigma_{e1}), log(\sigma_{e2}), a_0, a_1, a_2, a_3, \eta_{20}, a_4, a_5]\)
\item
  We now specify the model structure and control parameters\\
\item
  xstart, ub and lb contain, respectively, the initial values for the
  parameters, upper, and lower bounds
\end{itemize}

\begin{verbatim}
model <- dynr.model(
              num_regime=2,
              dim_latent_var=2,
              xstart=c(rep(log(.1), 4), log(10.0), log(10.0), -3.0, 9.0, -1.5, -0.5, 95.0,-.3,-.3),
              ub=c(rep(10, 6), rep(20, 4), 1000, 20, 20),
              lb=c(rep(-10, 6), rep(-20, 4), 0, -20, -20),
              options=list(maxtime=1*60, maxeval=20)
)
\end{verbatim}

\end{frame}

\begin{frame}[fragile]{Running the model}

\begin{verbatim}
res <- dynr.run(model, data)
summary(res)
\end{verbatim}

\end{frame}

\begin{frame}[fragile]{Some default dynr plots (work in-progress)}

\begin{verbatim}
plot(res, data=data, graphingPar=list(cex.main=1, cex.axis=1, cex.lab=1.2), numSubjDemo=2)

dynr.ggplot(res, data.dynr=data, states=c(1,2), names.state=paste0("state",states), title="Smoothed State Values", numSubjDemo=2)
\end{verbatim}

\end{frame}
